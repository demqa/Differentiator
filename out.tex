\input{header.tex}     
\usepackage{upgreek} 

\begin{document}     



Пошёл ты нахуй, мусор, Я drum n' bass продюсер


\begin{dmath}
({{{ \sin  x } ^ { \cos ({ 6 } \cdot {{ x } ^ { \sh  x }})}} \cdot {({{ 3 } \cdot {{ x } ^ { 2 }}} - { \cfrac {{ e } ^ { x }}{ x }})}} + { 1 })' = ({{ \sin  x } ^ { \cos ({ 6 } \cdot {{ x } ^ { \sh  x }})}} \cdot {({{ 3 } \cdot {{ x } ^ { 2 }}} - { \cfrac {{ e } ^ { x }}{ x }})})' + ( 1 )'
\end{dmath}


Я курю, и мне похуй, я бухаю, и мне похуй, Жру таблетки, и мне похуй. Трахнул суку без гондона, и мне похуй, Двадцать тысяч на кроссовки, Трачу деньги на хуйню и трачу их без остановки


\begin{dmath}
({{ \sin  x } ^ { \cos ({ 6 } \cdot {{ x } ^ { \sh  x }})}} \cdot {({{ 3 } \cdot {{ x } ^ { 2 }}} - { \cfrac {{ e } ^ { x }}{ x }})})' = ({ \sin  x } ^ { \cos ({ 6 } \cdot {{ x } ^ { \sh  x }})})' \cdot ({{ 3 } \cdot {{ x } ^ { 2 }}} - { \cfrac {{ e } ^ { x }}{ x }}) + { \sin  x } ^ { \cos ({ 6 } \cdot {{ x } ^ { \sh  x }})} \cdot (({{ 3 } \cdot {{ x } ^ { 2 }}} - { \cfrac {{ e } ^ { x }}{ x }}))'
\end{dmath}


Ну так чего же мы сидим?


\begin{dmath}
({ \sin  x } ^ { \cos ({ 6 } \cdot {{ x } ^ { \sh  x }})})' = { \sin  x } ^ { \cos ({ 6 } \cdot {{ x } ^ { \sh  x }})} \cdot ( \ln  \sin  x  \cdot ( \cos ({ 6 } \cdot {{ x } ^ { \sh  x }}))' +  \cfrac {( \sin  x )' \cdot  \cos ({ 6 } \cdot {{ x } ^ { \sh  x }})}{ \sin  x })
\end{dmath}


Совершенно очевидно, что


\begin{dmath}
( \cos ({ 6 } \cdot {{ x } ^ { \sh  x }}))' =  -  \sin ({ 6 } \cdot {{ x } ^ { \sh  x }}) \cdot ({ 6 } \cdot {{ x } ^ { \sh  x }})'
\end{dmath}


Для любого эпсилон известно, что...


\begin{dmath}
({ 6 } \cdot {{ x } ^ { \sh  x }})' = ( 6 )' \cdot { x } ^ { \sh  x } +  6  \cdot ({ x } ^ { \sh  x })'
\end{dmath}


Совершенно очевидно, что


\begin{dmath}
( 6 )' = 0
\end{dmath}


Это всё понятно, а что более интересно, так это


\begin{dmath}
({ x } ^ { \sh  x })' = { x } ^ { \sh  x } \cdot ( \ln  x  \cdot ( \sh  x )' +  \cfrac {( x )' \cdot  \sh  x }{ x })
\end{dmath}


Ускорил прогу в 10 раз, моё имя - Рома Глаз!


\begin{dmath}
( \sh  x )' =  \ch  x  \cdot ( x )'
\end{dmath}


НЕ ВЗЯЛ ПРОИЗВОДНУЮ? Боже какооой пустяк... сделать хоть раз что-нибудь не так. Выкинуть хлам из дома, и старых позвать друзей...


\begin{dmath}
( x )' = 1
\end{dmath}


Пошёл ты нахуй, мусор, Я drum n' bass продюсер


\begin{dmath}
( x )' = 1
\end{dmath}


1000 - 7, всё в башке плывёт совсем, апнул снова новый ранг - но по итогу стал никем, стану лучше и сильней, в нике мод Канеки Кен, я самый мёртвый гуль на фоне этих озверевших тел (ха)


\begin{dmath}
( \sin  x )' =  \cos  x  \cdot ( x )'
\end{dmath}


Ну давай добавляй сюда чё-нибудь


\begin{dmath}
( x )' = 1
\end{dmath}


1000 - 7, всё в башке плывёт совсем, апнул снова новый ранг - но по итогу стал никем, стану лучше и сильней, в нике мод Канеки Кен, я самый мёртвый гуль на фоне этих озверевших тел (ха)


\begin{dmath}
({{ 3 } \cdot {{ x } ^ { 2 }}} - { \cfrac {{ e } ^ { x }}{ x }})' = ({ 3 } \cdot {{ x } ^ { 2 }})' - ( \cfrac {{ e } ^ { x }}{ x })'
\end{dmath}


Совершенно очевидно, что


\begin{dmath}
({ 3 } \cdot {{ x } ^ { 2 }})' = ( 3 )' \cdot { x } ^ { 2 } +  3  \cdot ({ x } ^ { 2 })'
\end{dmath}


Для любого эпсилон известно, что...


\begin{dmath}
( 3 )' = 0
\end{dmath}


А ведь папа говорил, что нужно учиться...


\begin{dmath}
({ x } ^ { 2 })' = {( 2  - 1) \cdot  x  ^ { 2  - 1} \cdot ( x )'}
\end{dmath}


Сим-салабим


\begin{dmath}
( x )' = 1
\end{dmath}


А пошли они на хутор, бабочек ловить.


\begin{dmath}
( \cfrac {{ e } ^ { x }}{ x })' =  \cfrac {({ e } ^ { x })' \cdot  x  - { e } ^ { x } \cdot ( x )'}{{ x } ^ 2}
\end{dmath}


Я курю, и мне похуй, я бухаю, и мне похуй, Жру таблетки, и мне похуй. Трахнул суку без гондона, и мне похуй, Двадцать тысяч на кроссовки, Трачу деньги на хуйню и трачу их без остановки


\begin{dmath}
({ e } ^ { x })' = { e } ^ { x } \cdot  \ln  e  \cdot ( x )'
\end{dmath}


Это всё понятно, а что более интересно, так это


\begin{dmath}
( x )' = 1
\end{dmath}


Как говорят американцы: кто много знает, тот пули глотает!


\begin{dmath}
( x )' = 1
\end{dmath}


Об этом даже не стоит упоминать, но


\begin{dmath}
( 1 )' = 0
\end{dmath}
В общем
\begin{dmath}
({{{ \sin  x } ^ { \cos ({ 6 } \cdot {{ x } ^ { \sh  x }})}} \cdot {({{ 3 } \cdot {{ x } ^ { 2 }}} - { \cfrac {{ e } ^ { x }}{ x }})}} + { 1 })' = {{{{ \sin  x } ^ { \cos ({ 6 } \cdot {{ x } ^ { \sh  x }})}} \cdot {({{{({} - { \sin ({ 6 } \cdot {{ x } ^ { \sh  x }})})} \cdot {{ 6 } \cdot {{{ x } ^ { \sh  x }} \cdot {({{ \ch  x } \cdot { \ln  x }} + { \cfrac { \sh  x }{ x }})}}}} \cdot { \ln ( \sin  x )}} + { \cfrac {{ \cos  x } \cdot { \cos ({ 6 } \cdot {{ x } ^ { \sh  x }})}}{ \sin  x }})}} \cdot {({{ 3 } \cdot {{ x } ^ { 2 }}} - { \cfrac {{ e } ^ { x }}{ x }})}} + {{{ \sin  x } ^ { \cos ({ 6 } \cdot {{ x } ^ { \sh  x }})}} \cdot {({{ 3 } \cdot {{ 2 } \cdot { x }}} - { \cfrac {({{{ e } ^ { x }} \cdot { x }} - {{ e } ^ { x }})}{{ x } ^ { 2 }}})}}
\end{dmath}


\end{document}