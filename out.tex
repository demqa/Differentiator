%%% Класс документа
\documentclass[a4paper,10pt]{article}

%%% Работа с русским языком
\usepackage{cmap}					% поиск в PDF
\usepackage[warn]{mathtext}
\usepackage[T2A]{fontenc}			% кодировка
\usepackage[utf8]{inputenc}			% кодировка исходного текста
\usepackage[english,russian]{babel}	% локализация и переносы
\usepackage{mathtext} 				% русские буквы в формулах
\usepackage{csvsimple}              % for tabular from csv loading
\usepackage{indentfirst}            % indent after sections
%\usepackage{minipage}

%%% Дополнительная работа с математикой
\usepackage{amsmath,amsfonts,amssymb,amsthm,mathtools} % AMS
%\usepackage{icomma} % "Умная" запятая: $0,2$ --- число, $0, 2$ --- перечисление
\usepackage{pdflscape}
\usepackage{breqn}

%%% Номера формул
%\mathtoolsset{showonlyrefs=true} % Показывать номера только у тех формул, на которые есть \eqref{} в тексте.
%\usepackage{leqno} % Немуреация формул слева

%%% Шрифты
\usepackage{euscript}	 % Шрифт Евклид
\usepackage{mathrsfs} % Красивый матшрифт

%%% Свои команды
\DeclareMathOperator{\sgn}{\mathop{sgn}}

%%% Перенос знаков в формулах (по Львовскому)
\newcommand*{\hm}[1]{#1\nobreak\discretionary{}
{\hbox{$\mathsurround=0pt #1$}}{}}

%%% Работа с картинками
\usepackage{graphicx}  % Для вставки рисунков
\graphicspath{{images/}{images2/}}  % папки с картинками
\setlength\fboxsep{3pt} % Отступ рамки \fbox{} от рисунка
\setlength\fboxrule{1pt} % Толщина линий рамки \fbox{}
\usepackage{wrapfig} % Обтекание рисунков и таблиц текстом

%%% Работа с таблицами
\usepackage{array,tabularx,tabulary,booktabs} % Дополнительная работа с таблицами
\usepackage{longtable}  % Длинные таблицы
\usepackage{multirow} % Слияние строк в таблице

%%% Теоремы
\theoremstyle{plain} % Это стиль по умолчанию, его можно не переопределять.
%\newtheorem{theorem}{Теорема}[section]
%\newtheorem{proposition}[theorem]{Утверждение}
 
%\theoremstyle{definition} % "Определение"
%\newtheorem{corollary}{Следствие}[theorem]
%\newtheorem{problem}{Задача}[section]
 
%\theoremstyle{remark} % "Примечание"
%\newtheorem*{nonum}{Решение}

%%% Программирование
\usepackage{etoolbox} % логические операторы

%%% Страница
\usepackage{extsizes} % Возможность сделать 14-й шрифт
\usepackage{geometry} % Простой способ задавать поля
	\geometry{top=25mm}
	\geometry{bottom=35mm}
	\geometry{left=10mm}
	\geometry{right=10mm}
	
%%% Колонтитулы
%\usepackage{fancyhdr}
 	%\pagestyle{fancy}
 	%\renewcommand{\headrulewidth}{0mm}  % Толщина линейки, отчеркивающей верхний колонтитул
 	%\lfoot{Нижний левый}
 	%\rfoot{Нижний правый}
 	%\rhead{Верхний правый}
 	%\chead{Верхний в центре}
 	%\lhead{Верхний левый}
 	% \cfoot{Нижний в центре} % По умолчанию здесь номер страницы
 	
%%% Интерлиньяж
%\usepackage{setspace}
%\onehalfspacing % Интерлиньяж 1.5
%\doublespacing % Интерлиньяж 2
%\singlespacing % Интерлиньяж 1

%%% Гиперссылки
\usepackage{hyperref}
\usepackage[usenames,dvipsnames,svgnames,table,rgb]{xcolor}
\hypersetup{				% Гиперссылки
    unicode=true,           % русские буквы в раздела PDF
    pdftitle={Заголовок},   % Заголовок
    pdfauthor={Автор},      % Автор
    pdfsubject={Тема},      % Тема
    pdfcreator={Создатель}, % Создатель
    pdfproducer={Производитель}, % Производитель
    pdfkeywords={keyword1} {key2} {key3}, % Ключевые слова
    colorlinks=true,       	% false: ссылки в рамках; true: цветные ссылки
    linkcolor=red,          % внутренние ссылки
    citecolor=green,        % на библиографию
    filecolor=magenta,      % на файлы
    urlcolor=Mulberry          % на URL
}

%%% Другие пакеты
\usepackage{lastpage} % Узнать, сколько всего страниц в документе.
\usepackage{soul} % Модификаторы начертания
\usepackage{csquotes} % Еще инструменты для ссылок
%\usepackage[style=authoryear,maxcitenames=2,backend=biber,sorting=nty]{biblatex}
\usepackage{multicol} % Несколько колонок
\usepackage{multirow} % Несколько строк

%%% Шрифты
%\renewcommand{\familydefault}{\sfdefault} % Начертание шрифта


%%% Работа с библиографией
%\usepackage{cite} % Работа с библиографией
%\usepackage[superscript]{cite} % Ссылки в верхних индексах
%\usepackage[nocompress]{cite} % 
%\usepackage{csquotes} % Еще инструменты для ссылок


%%% Tikz
\usepackage{tikz} % Работа с графикой
\usepackage{pgfplots} % Работа с pgf
\usepackage{pgfplotstable}

\definecolor{linkcolor}{HTML}{799B03} % цвет ссылок
\definecolor{urlcolor}{HTML}{799B03} % цвет гиперссылок

%%% Дополнительные пакеты для tikz
\usepgfplotslibrary{dateplot} % Возможность подписания дат
\pgfplotsset{compat=1.5}     
\usepackage{upgreek} 

\begin{document}     



Пошёл ты нахуй, мусор, Я drum n' bass продюсер


\begin{dmath}
({{{ \sin  x } ^ { \cos ({ 6 } \cdot {{ x } ^ { \sh  x }})}} \cdot {({{ 3 } \cdot {{ x } ^ { 2 }}} - { \cfrac {{ e } ^ { x }}{ x }})}} + { 1 })' = ({{ \sin  x } ^ { \cos ({ 6 } \cdot {{ x } ^ { \sh  x }})}} \cdot {({{ 3 } \cdot {{ x } ^ { 2 }}} - { \cfrac {{ e } ^ { x }}{ x }})})' + ( 1 )'
\end{dmath}


Я курю, и мне похуй, я бухаю, и мне похуй, Жру таблетки, и мне похуй. Трахнул суку без гондона, и мне похуй, Двадцать тысяч на кроссовки, Трачу деньги на хуйню и трачу их без остановки


\begin{dmath}
({{ \sin  x } ^ { \cos ({ 6 } \cdot {{ x } ^ { \sh  x }})}} \cdot {({{ 3 } \cdot {{ x } ^ { 2 }}} - { \cfrac {{ e } ^ { x }}{ x }})})' = ({ \sin  x } ^ { \cos ({ 6 } \cdot {{ x } ^ { \sh  x }})})' \cdot ({{ 3 } \cdot {{ x } ^ { 2 }}} - { \cfrac {{ e } ^ { x }}{ x }}) + { \sin  x } ^ { \cos ({ 6 } \cdot {{ x } ^ { \sh  x }})} \cdot (({{ 3 } \cdot {{ x } ^ { 2 }}} - { \cfrac {{ e } ^ { x }}{ x }}))'
\end{dmath}


Ну так чего же мы сидим?


\begin{dmath}
({ \sin  x } ^ { \cos ({ 6 } \cdot {{ x } ^ { \sh  x }})})' = { \sin  x } ^ { \cos ({ 6 } \cdot {{ x } ^ { \sh  x }})} \cdot ( \ln  \sin  x  \cdot ( \cos ({ 6 } \cdot {{ x } ^ { \sh  x }}))' +  \cfrac {( \sin  x )' \cdot  \cos ({ 6 } \cdot {{ x } ^ { \sh  x }})}{ \sin  x })
\end{dmath}


Совершенно очевидно, что


\begin{dmath}
( \cos ({ 6 } \cdot {{ x } ^ { \sh  x }}))' =  -  \sin ({ 6 } \cdot {{ x } ^ { \sh  x }}) \cdot ({ 6 } \cdot {{ x } ^ { \sh  x }})'
\end{dmath}


Для любого эпсилон известно, что...


\begin{dmath}
({ 6 } \cdot {{ x } ^ { \sh  x }})' = ( 6 )' \cdot { x } ^ { \sh  x } +  6  \cdot ({ x } ^ { \sh  x })'
\end{dmath}


Совершенно очевидно, что


\begin{dmath}
( 6 )' = 0
\end{dmath}


Это всё понятно, а что более интересно, так это


\begin{dmath}
({ x } ^ { \sh  x })' = { x } ^ { \sh  x } \cdot ( \ln  x  \cdot ( \sh  x )' +  \cfrac {( x )' \cdot  \sh  x }{ x })
\end{dmath}


Ускорил прогу в 10 раз, моё имя - Рома Глаз!


\begin{dmath}
( \sh  x )' =  \ch  x  \cdot ( x )'
\end{dmath}


НЕ ВЗЯЛ ПРОИЗВОДНУЮ? Боже какооой пустяк... сделать хоть раз что-нибудь не так. Выкинуть хлам из дома, и старых позвать друзей...


\begin{dmath}
( x )' = 1
\end{dmath}


Пошёл ты нахуй, мусор, Я drum n' bass продюсер


\begin{dmath}
( x )' = 1
\end{dmath}


1000 - 7, всё в башке плывёт совсем, апнул снова новый ранг - но по итогу стал никем, стану лучше и сильней, в нике мод Канеки Кен, я самый мёртвый гуль на фоне этих озверевших тел (ха)


\begin{dmath}
( \sin  x )' =  \cos  x  \cdot ( x )'
\end{dmath}


Ну давай добавляй сюда чё-нибудь


\begin{dmath}
( x )' = 1
\end{dmath}


1000 - 7, всё в башке плывёт совсем, апнул снова новый ранг - но по итогу стал никем, стану лучше и сильней, в нике мод Канеки Кен, я самый мёртвый гуль на фоне этих озверевших тел (ха)


\begin{dmath}
({{ 3 } \cdot {{ x } ^ { 2 }}} - { \cfrac {{ e } ^ { x }}{ x }})' = ({ 3 } \cdot {{ x } ^ { 2 }})' - ( \cfrac {{ e } ^ { x }}{ x })'
\end{dmath}


Совершенно очевидно, что


\begin{dmath}
({ 3 } \cdot {{ x } ^ { 2 }})' = ( 3 )' \cdot { x } ^ { 2 } +  3  \cdot ({ x } ^ { 2 })'
\end{dmath}


Для любого эпсилон известно, что...


\begin{dmath}
( 3 )' = 0
\end{dmath}


А ведь папа говорил, что нужно учиться...


\begin{dmath}
({ x } ^ { 2 })' = {( 2  - 1) \cdot  x  ^ { 2  - 1} \cdot ( x )'}
\end{dmath}


Сим-салабим


\begin{dmath}
( x )' = 1
\end{dmath}


А пошли они на хутор, бабочек ловить.


\begin{dmath}
( \cfrac {{ e } ^ { x }}{ x })' =  \cfrac {({ e } ^ { x })' \cdot  x  - { e } ^ { x } \cdot ( x )'}{{ x } ^ 2}
\end{dmath}


Я курю, и мне похуй, я бухаю, и мне похуй, Жру таблетки, и мне похуй. Трахнул суку без гондона, и мне похуй, Двадцать тысяч на кроссовки, Трачу деньги на хуйню и трачу их без остановки


\begin{dmath}
({ e } ^ { x })' = { e } ^ { x } \cdot  \ln  e  \cdot ( x )'
\end{dmath}


Это всё понятно, а что более интересно, так это


\begin{dmath}
( x )' = 1
\end{dmath}


Как говорят американцы: кто много знает, тот пули глотает!


\begin{dmath}
( x )' = 1
\end{dmath}


Об этом даже не стоит упоминать, но


\begin{dmath}
( 1 )' = 0
\end{dmath}
В общем
\begin{dmath}
({{{ \sin  x } ^ { \cos ({ 6 } \cdot {{ x } ^ { \sh  x }})}} \cdot {({{ 3 } \cdot {{ x } ^ { 2 }}} - { \cfrac {{ e } ^ { x }}{ x }})}} + { 1 })' = {{{{{ \sin  x } ^ { \cos ({ 6 } \cdot {{ x } ^ { \sh  x }})}} \cdot {({{{({} - { \sin ({ 6 } \cdot {{ x } ^ { \sh  x }})})} \cdot {({{ 0 } \cdot {{ x } ^ { \sh  x }}} + {{ 6 } \cdot {{{ x } ^ { \sh  x }} \cdot {({{{ \ch  x } \cdot { 1 }} \cdot { \ln  x }} + { \cfrac {{ 1 } \cdot { \sh  x }}{ x }})}}})}} \cdot { \ln ( \sin  x )}} + { \cfrac {{{ \cos  x } \cdot { 1 }} \cdot { \cos ({ 6 } \cdot {{ x } ^ { \sh  x }})}}{ \sin  x }})}} \cdot {({{ 3 } \cdot {{ x } ^ { 2 }}} - { \cfrac {{ e } ^ { x }}{ x }})}} + {{{ \sin  x } ^ { \cos ({ 6 } \cdot {{ x } ^ { \sh  x }})}} \cdot {({{{ 0 } \cdot {{ x } ^ { 2 }}} + {{ 3 } \cdot {{{ 2 } \cdot {{ x } ^ { 1 }}} \cdot { 1 }}}} - { \cfrac {({{{{ e } ^ { x }} \cdot {{ \ln  e } \cdot { 1 }}} \cdot { x }} - {{{ e } ^ { x }} \cdot { 1 }})}{{ x } ^ { 2 }}})}}} + { 0 }
\end{dmath}


\end{document}